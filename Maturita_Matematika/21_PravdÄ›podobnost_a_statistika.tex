\title{ 22. Pravděpodobnost a statistika}
\author{Marek Fuchs}
\date{26.4.2025}

\maketitle



\section{Pravděpodobnost a statistika}
    \subsection{Náhodný pokus}
    V počtu pravděpodobnosti se jím rozumí libovolná opakovaná činnost za danných podmínek, jejíž výsledky závisí do jisté míry na náhodě.\\
    Např. hod mincí, kostek, míchání karet.
    \subsection{Množina všech možných výsledků náhodného pokusu $m$}
    Jedná se o množinu všech výsledků, které mohou nastat v uvažovaném náhodném pokusu.
    \subsection{Možné výsledky náhodného pokusu}
    \begin{itemize}
        \item Jsou navzájem rozlišitelné a v případě konečné množiny $m$ je lze vyjmenovat
        \item navzájem se vylučují tj. žádné dva nemůžou nastat současně
        \item 1 z nich vždy nastane tj. nemůže nastat výsledek, který by nepatřil do množiny
        \item jsou dány volbou množiny $m$
    \end{itemize}
    \subsection{Pravděpodobnost}
    Určuje chování náhodných jevů
    \begin{itemize}
            \item Jaká je šance, že daný jev nastane
            \item Podíl všech příznivých jevů $m(A)$ ku všem možným jevům $m$\\
            $$
            \frac{m(A)}{m}
            $$
            \item Nejzákladnější zápis pravděpodobnosti je zlomek
            \item Tento zlomek nemůže být větší jak jedna
        \end{itemize}
    \subsection{Typy jevů}
    \begin{itemize}
            \item $x=1$ - jev jistý (100\%)
            \item $x=0$ - jev nemožný, zapisuje se prázdou množinou (0\%)
            \item Doplňkový jev - Doplňkový jev k jevu $A$ je jev $A$´, který obsahuje všechny možné výsledky, které mohou nastat, ale nejsou zahrnuty v jevu $A$
            \item Sjednocení jevů $A$ a $B$ - zapisuje se $A\cup B$ a nastává tehdy, když nastane alespoň jeden z jevů $A$ nebo $B$
            \item Průnik jevů $A\cap B$ - Nastává tehdy, když nastanou oba jevy $A$ a $B$
        \end{itemize}
    \subsection{Pravděpodobnost jevu $A$}
    Poměr počtu $m(A)$ výsledků příznivých jevu $A$ a počtu $m$ všech možných výsledků náhodného jevu.\\
    $$
    P(A)=\frac{m(A)}{m},\space\space P(A) \in \ \langle0;1\rangle.
    $$
    Předpokládáme, že všechny výsledky náhodného jevu jsou stejně možné - při hodu více mincemi musíme být schopni navzájem mince rozlišit, aby výsledky jevů byly stejně možné.
    \subsection{Pravděpodobnost průniku jevů}
    $$
    P(A\cap B) = P(A) \cdot P(B),
    $$ 
    za předpokladu, že jevy jsou vzájemně nezávislé.
    \subsection{Pravděpodobnost sjednocení jevů}
    $$
    A \cap B=\emptyset; \  P(A\cup B)=P(A)+P(B),
    $$
    $$
    A \cap B\neq\emptyset; \  P(A\cup B)=P(A)+P(B)-P(A\cap B).
    $$
    \subsection{Variace}
    Variace k-té třídy z $n$ prvků je každá uspořádaná k-tice vytvořená z celkového počtu $n$ prvků, přičemž při výběru záleží na pořadí jednotlivých prvků.\\
    $$
    V_k (n)=\frac{n!}{(n-k)!}
    $$
    \subsection{Kombinace}
    Kombinace je neuspořádaná k-tice vytvořená z celkového počtu $n$ prvků, přičemž nezáleží na pořadí vybraných prvků
    $$
    C_k (n)= {n \choose k}.
    $$
    \subsection{Permutace}
    Permutace je zvláštní případ variace, kde $k=n$. To znamená, že ze zadaných prvků postupně vybereme všechny. Každá permutace tedy odpovídá nějakému pořadí zadaných prvků: každý prvek se v pořadí musí objevit, ale žádný tam nemůže být dvakrát. Permutace z $n$ prvků je každá n-členná variace z těchto prvků.
    $$
    P(n)=n!,
    $$
    $$
    P(n)=n \cdot(n-1)\cdot(n-2) \ ...2\cdot1=n!.
    $$
    \subsection{Kombinační číslo}
    Z prvkové množiny $n$ vybíráme z podprvkové množiny $k$\\
    $$
    {n \choose k}.
    $$
    \subsection{Kombinatorické pravidlo o součinu}
    Máme $a$ způsobů, jak něco provést, a $b$ způsobů jak udělat něco jiného, a je možné dělat obojí zároveň, pak existuje $a\cdot b$ způsobů jak danou činnost provést.
    \subsection{Kombinatorické pravidlo o součtu}
    Máme $a$ způsobů, jak něco provést, a $b$ způsobů jak udělat něco jiného, a je nemožné dělat obojí zároveň, pak pak existuje $a+b$ způsobů jak danou činnost provést.
    \subsection{Statistika}
    Vědní obor, zabývající se metodami kvantitativního hodnocení vlastností hromadných jevů a procesů.
    \subsection{Statistický soubor}
    Jistý konečný soubor zkoumaných dat.\\
    Počet prvků v statistickém souboru nazýváme $rozstah$ $souboru$.
    \subsection{Statistická jednotka}
    Konkrétní prvek statistického souboru.
    \subsection{Statistický znak}
    Znak, který chceme měřit.
    \subsection{Četnost}
    Četnost může být buď relativní nebo absolutní a udává, kolik hodnot daného znaku se vyskytuje ve statistickém souboru — buď absolutně, nebo relativně vzhledem k celkovému počtu prvků souboru.\\
    V množině se nemohou opakovat prvky.
    \subsection{Aritmetický průměr}
    Průměr všech hodnot ve statistickém souboru. Tedy podíl součtu všech hodnot ku počtu hodnot.\\
    $$
    p_a=\frac{x_1+x_2+x_3+x_n}{n}
    $$
    \subsection{Modus a medián}
    \begin{itemize}
        \item Modus - Hodnota, která má nejvyšší četnost, značí se $Mod(x)$.
        \item Medián - Prostřední hodnota, značíme $Med(x)$.
    \end{itemize}
    