\title{6 Exponenciální a logaritmické rovnice a nerovnice}
\author{Roman Šnajder}
\date{April 2025}

\maketitle

\section{Exponenciální a logaritmické rovnice a nerovnice}
\subsection {Exponenciální rovnice}

Z běžné rovnice se exponenciální stává, pokud obsahuje proměnnou v exponentu. Typickým příkladem exponenciální rovnice může být třeba $2^x = 8$.
\subsection {Pravidla pro počítání s exponenty}
$$a^x\cdot a^y=a^{x+y}$$
$$a^{-x}=\frac{1}{a^x}$$
$$a^{x^{y}}=a^{xy}$$
$$a^{x-y}=\frac{a^x}{a^y}$$
$$a^0=1$$
Rovnice typu $a^{f(x)}=a^{g(x)}$ se řeší porovnáním exponentů (pro $a>0$, a $\neq 1$). Rovnice typu $a^{f(x)}=b^{g(x)}$, se řeší logaritmováním na tvar $f(x)\cdot\log_a a=g(x)\cdot\log_a b$ (pro $a>0$, $b>0$, $a\neq1$, $b\neq1$)
\subsubsection{Jednoduchá exponenciální rovnice}
Při řešení exponenciální rovnici, je žádoucí, pokud lze rovnici upravit na tvar o stejném základu na obou stranách rovnice.
$$2^3=8$$
$$2^x=2^3$$
$$x=3$$
Další případ nastává, pokud se na jedné straně rovnice vyskytuje jednička. Základ obsahující číslo 1 je možné zapsat jako libovolné číslo na nultou a tím vznikne exponenciální rovnice o stejném základu.
$$5^x=1$$
$$5^x=5^0$$
$$x=0$$
V exponenciální rovnici lze využít i substituce, kdy členy s neznámou jsou nahrazeny písmenem, které však po vypočtení nově vzniklé rovnice musí být dosazeno zpět do původního výrazu a až tento výsledek je řešením dané rovnice. Při použití této metody často vznikne kvadratická rovnice.
$$9^x-25\cdot3^x-54=0$$
$$3^{2x}-25\cdot3^x-54=0$$
$$t^2-25t-54=0$$
$$(t+2)\cdot(t-27)=0$$
$$3^x=-2,x_1\neq R$$
$$3^x=27$$
$$x_2=3$$
\subsection {Exponenciální nerovnice}
Exponenciální nerovnice se řeší stejně jako rovnice. Liší se pouze u výsledku, kdy u nerovnice vznikne interval nebo intervaly. Pokud je neznámá menší než 0, pak se nerovnost vynásobí (-1).
$$2^{1-x}>\frac{1}{2}$$
$$2^{1-x}>2^{-1}$$
$$1-x>-1$$
$$-x>-2 /\cdot(-1)$$
$$x<2$$
$$K=(-\infty;2)$$
Stejně jako u rovnice i zde lze použít substituce s tím rozdílem, že řešením bude interval nebo intervaly, pro které má daná rovnice řešení.
$$9^x-25\cdot3^x-54>0$$
$$3^{2x}-25\cdot3^x-54>0$$
$$t^2-25t-54>0$$
$$(t+2)\cdot(t-27)>0$$
$$\begin{tabular}{c|c|c}
    - & + & + \\
    - & - & + 
\end{tabular}$$
$$x=(-\infty;-2)\cup(27;\infty)$$
\subsection {Pravidla pro počítání s logaritmy}
$$\log_{a}x+\log _{a}y=\log_{a}(xy)$$
$$\log_{a}x-\log_{a}y=\log_{a}\frac{x}{y}$$
$$\log_{a}x^y=y\cdot\log_{a}x$$
$$\log_{a}a=1$$
$$\log_{a}x=\frac{\log_c x}{\log_c a}$$
\subsubsection{Logaritmické rovnice}
Základní logaritmická rovnice s užitím pravidla. Zde však musí být určeny podmínky, jelikož logaritmus nemůže být záporný a tak budu řešit pouze pro x, která nám vyjdou kladná. (V tomto případě čísla $x \in (0;\infty)$)
$$\log_{2}x=3$$
$$x=2^3$$
$$x=8$$
Pokud je v příkladu logaritmus a k tomu nějaké další číslo, pak z tohoto čísla vytvoříme logaritmus a příklad se vyřeší podobně jako ten předchozí. Opět musí být určen definiční obor $$(1;\infty)$$\\
$$\log_{2}(x-1)+2=\log_{2}(2x+1)$$
$$\log_{2}(x-1)+\log_{2}4=\log_{2}(2x+1)$$
$$\log_{2}[(x-1)\cdot4]=\log_{2}(2x+1)$$
$$4x-4=2x+1$$
$$x=\frac{5}{2}$$
U logaritmický nerovnice opět použít substituce.
$$\log_{2}^2x-2log_{2}x+1=0$$
$$\log_{2}x=t$$
$$t^2-2t+1=0$$
$$(t-1)^2=0$$
$$\log_{2}x=1, x=2$$
\subsubsection{Logaritmické nerovnice}
Logaritmické nerovnice se řeší stejně jako rovnice. Liší se pouze u výsledku, kdy u nerovnice vznikne interval nebo intervaly. I u nerovnic lze použít substituce. Pokud je základ logaritmu menší než 1, pak se nerovnost musí otočit! Dalším důležitým krokem je si, tak jako u rovnic, určit definiční obor.
$$\log_{\frac{1}{2}}(2-x)<-3$$
$$\log_{\frac{1}{2}}(2-x)<\log_{\frac{1}{2}}8$$
$$2-x<8$$
$$x<6$$
$$x \in(-\infty;6)$$
Z -3 se stane $\log_{\frac{1}{2}}8$ díky pravidlu pro počítání s logaritmy.

\subsection {Použití logaritmu u exponenciálních rovnic a nerovnic}
U rovnic i nerovnic se může sát, že žádnou jednoduchou úpravou nelze dostat stejné základy. Zde však lze zlogaritmovat obě strany logaritmem o stejném základu. Řešení je stejné jak pro rovnice tak pro nerovnice, liší se pouze výsledkem. U nerovnic to bude interval a u rovnic nějaké konkrétní číslo.
$$2^x>5$$
$$x\cdot\log_{2}2>\log_{2}5$$
$$x>\log_{2}5$$